%% Generated by Sphinx.
\def\sphinxdocclass{report}
\documentclass[letterpaper,10pt,english]{sphinxmanual}
\ifdefined\pdfpxdimen
   \let\sphinxpxdimen\pdfpxdimen\else\newdimen\sphinxpxdimen
\fi \sphinxpxdimen=.75bp\relax
\ifdefined\pdfimageresolution
    \pdfimageresolution= \numexpr \dimexpr1in\relax/\sphinxpxdimen\relax
\fi
%% let collapsible pdf bookmarks panel have high depth per default
\PassOptionsToPackage{bookmarksdepth=5}{hyperref}

\PassOptionsToPackage{booktabs}{sphinx}
\PassOptionsToPackage{colorrows}{sphinx}

\PassOptionsToPackage{warn}{textcomp}
\usepackage[utf8]{inputenc}
\ifdefined\DeclareUnicodeCharacter
% support both utf8 and utf8x syntaxes
  \ifdefined\DeclareUnicodeCharacterAsOptional
    \def\sphinxDUC#1{\DeclareUnicodeCharacter{"#1}}
  \else
    \let\sphinxDUC\DeclareUnicodeCharacter
  \fi
  \sphinxDUC{00A0}{\nobreakspace}
  \sphinxDUC{2500}{\sphinxunichar{2500}}
  \sphinxDUC{2502}{\sphinxunichar{2502}}
  \sphinxDUC{2514}{\sphinxunichar{2514}}
  \sphinxDUC{251C}{\sphinxunichar{251C}}
  \sphinxDUC{2572}{\textbackslash}
\fi
\usepackage{cmap}
\usepackage[T1]{fontenc}
\usepackage{amsmath,amssymb,amstext}
\usepackage{babel}



\usepackage{tgtermes}
\usepackage{tgheros}
\renewcommand{\ttdefault}{txtt}



\usepackage[Bjarne]{fncychap}
\usepackage[,numfigreset=1,mathnumfig]{sphinx}

\fvset{fontsize=auto}
\usepackage{geometry}


% Include hyperref last.
\usepackage{hyperref}
% Fix anchor placement for figures with captions.
\usepackage{hypcap}% it must be loaded after hyperref.
% Set up styles of URL: it should be placed after hyperref.
\urlstyle{same}

\addto\captionsenglish{\renewcommand{\contentsname}{User Manual}}

\usepackage{sphinxmessages}
\setcounter{tocdepth}{1}


\makeatletter
   \fancypagestyle{normal}{
% this is the stuff in sphinx.sty
    \fancyhf{}
    \fancyfoot[LE,RO]{{\py@HeaderFamily\thepage}}
% we comment this out and
    %\fancyfoot[LO]{{\py@HeaderFamily\nouppercase{\rightmark}}}
    %\fancyfoot[RE]{{\py@HeaderFamily\nouppercase{\leftmark}}}
% add copyright stuff
    \fancyfoot[LO,RE]{{\textcopyright\ 2023, us4us Ltd.}}
% again original stuff
    \fancyhead[LE,RO]{{\py@HeaderFamily \@title\sphinxheadercomma\py@release}}
    \renewcommand{\headrulewidth}{0.4pt}
    \renewcommand{\footrulewidth}{0.4pt}
    }
% this is applied to each opening page of a chapter
   \fancypagestyle{plain}{
    \fancyhf{}
    \fancyfoot[LE,RO]{{\py@HeaderFamily\thepage}}
    \renewcommand{\headrulewidth}{0pt}
    \renewcommand{\footrulewidth}{0.4pt}
% add copyright stuff for example at left of footer on odd pages,
% which is the case for chapter opening page by default
    \fancyfoot[LO,RE]{{\textcopyright\ 2023, us4us Ltd.}}
    }
\makeatother


\title{Us4R User Manual}
\date{Dec 12, 2023}
\release{}
\author{us4us Ltd}
\newcommand{\sphinxlogo}{\vbox{}}
\renewcommand{\releasename}{}
\makeindex
\begin{document}

\ifdefined\shorthandoff
  \ifnum\catcode`\=\string=\active\shorthandoff{=}\fi
  \ifnum\catcode`\"=\active\shorthandoff{"}\fi
\fi

\pagestyle{empty}
\sphinxmaketitle
\pagestyle{plain}
\sphinxtableofcontents
\pagestyle{normal}
\phantomsection\label{\detokenize{index::doc}}


\sphinxstepscope


\chapter{Intended Use}
\label{\detokenize{content/intro:intended-use}}\label{\detokenize{content/intro::doc}}
\sphinxAtStartPar
The Advanced Ultrasound Research Platform \sphinxstylestrong{(us4R™)} is an ultrasonic
system intended to be used in an uncontrolled laboratory setting for
ultrasound R\&D, in particular for real\sphinxhyphen{}time implementation of new
imaging modalities and algorithms in biomedical and non\sphinxhyphen{}destructive
testing applications. \sphinxstylestrong{The system is not a medical device and is not
intended for use on humans.}

\sphinxAtStartPar
The \sphinxstylestrong{us4R™} is a fully programmable ultrasound device built on a novel
architecture optimized for streaming acquisition and software processing
of raw RF echo signals with the help of GPUs.

\begin{sphinxadmonition}{attention}{Attention:}
\sphinxAtStartPar
The device can only be operated by users with a base knowledge of programming and fundamental PC skills. It is essential that users read the full text of the instruction manual before operating the device.
\end{sphinxadmonition}


\chapter{Interaction of ultrasound}
\label{\detokenize{content/intro:interaction-of-ultrasound}}
\sphinxAtStartPar
Ultrasonic waves are mechanical waves, which propagate in a medium (such
as liquid, gas, solid and biological tissue). The propagation of
ultrasonic waves is related to the transport of energy.


\section{Phenomena accompanying the ultrasound wave}
\label{\detokenize{content/intro:phenomena-accompanying-the-ultrasound-wave}}
\sphinxAtStartPar
The intensity of the ultrasound wave (the maximum pressure value)
generated by the \sphinxstylestrong{us4R™} should not constitute a hazard for test
subjects. Nevertheless, it is important to understand how ultrasound
interacts with tissues and realize the possible bioeffects caused by
mechanical waves.

\sphinxAtStartPar
These bio\sphinxhyphen{}effects fall within two categories: mechanical \textendash{} related to
the possibility of cavitation, and thermal \textendash{} related to the absorption
of wave energy by the tissue, which is converted into heat. The
amplitude of an ultrasound wave propagating in any absorbing medium,
such as a soft tissue, decreases with distance. Damping caused by the
absorption and dissipation of the wave leads to energy loss. In the
absorption process, part of the energy is converted into heat.

\sphinxAtStartPar
For a detailed examination of ultrasound safety, please consult \sphinxstyleemphasis{The Safe Use of Ultrasound in Medical Diagnosis}, 3rd ed., edited by Gail ter Haar.

\begin{sphinxadmonition}{attention}{Attention:}
\sphinxAtStartPar
The user creating a new transmit schema for the us4R™ should always consider the mentioned effects of the ultrasonic wave in a given medium/propagation environment.
\end{sphinxadmonition}

\sphinxstepscope


\chapter{Unboxing and setting\sphinxhyphen{}up the device}
\label{\detokenize{content/set-up:unboxing-and-setting-up-the-device}}\label{\detokenize{content/set-up::doc}}
\sphinxAtStartPar
The device is delivered to the user pre\sphinxhyphen{}assembled and boxed.

\sphinxAtStartPar
The user must connect the components and perform a setup before the
first use (see: {\hyperref[\detokenize{content/set-up:-system-setup}]{\sphinxcrossref{\DUrole{xref,myst}{5 System setup}}}})\sphinxstyleemphasis{.} The unboxing of the device should
be performed with utmost care.

\begin{sphinxadmonition}{caution}{Caution:}
\sphinxAtStartPar
If the device was placed in an environment with climatic conditions that sharply diverge from normal office conditions, it should undergo a process of acclimatization before the first use. This comprises of leaving the device out of the transport packaging for a minimum of 12 hours.
\end{sphinxadmonition}

\sphinxAtStartPar
The packaging should include:
\begin{itemize}
\item {} 
\sphinxAtStartPar
\sphinxstylestrong{us4R™} device (with probe adapter) \textendash{} 1 pcs.;

\item {} 
\sphinxAtStartPar
PCIe cables \textendash{} 4 or 8 pcs.;

\item {} 
\sphinxAtStartPar
PCIe adapter card \textendash{} 1 or 2 pcs.;

\item {} 
\sphinxAtStartPar
mains power cable \textendash{} 1 pcs.;

\item {} 
\sphinxAtStartPar
(optional) PC system controller \textendash{} 1 pcs.;

\item {} 
\sphinxAtStartPar
(optional) ultrasound probe(s).

\end{itemize}

\sphinxAtStartPar
In case of any missing items, the Customer is advised to contact the
Manufacturer.

\sphinxAtStartPar
Before the first use, it is necessary to ensure that the room has ample
space, stable ground and 120/230VAC mains power source with a protective
bonding.

\sphinxAtStartPar
The device should be placed to facilitate a safe operation: the power
cables must be neither strained nor hanging too loose in a manner that
may lead to tripping, wrenching out of cables or otherwise damaging them
through breaking or cutting.

\sphinxAtStartPar
Procedures using the \sphinxstylestrong{us4R™} should not be performed if the device is
in proximity to another working ultrasound device. Ultrasound probes can
cause interference, resulting in a falsification of the image.

\sphinxAtStartPar
A proper operation of the device is described in the next chapters of
this manual.

\begin{sphinxadmonition}{caution}{Caution:}
\sphinxAtStartPar
The power cables should be plugged into the 120\sphinxhyphen{}230VAC/50Hz/60Hz mains power supply with a protective bonding.
\end{sphinxadmonition}


\section{Probe Adapters}
\label{\detokenize{content/set-up:probe-adapters}}
\sphinxAtStartPar
Several adapters are available for use with the \sphinxstylestrong{us4R™} system. Please
consult the list of adapters as shown below:


\begin{savenotes}\sphinxattablestart
\sphinxthistablewithglobalstyle
\centering
\sphinxcapstartof{table}
\sphinxthecaptionisattop
\sphinxcaption{Probe adapters}\label{\detokenize{content/set-up:id1}}
\sphinxaftertopcaption
\begin{tabular}[t]{\X{15}{65}\X{20}{65}\X{30}{65}}
\sphinxtoprule
\sphinxstyletheadfamily 
\sphinxAtStartPar
Options *
&\sphinxstyletheadfamily 
\sphinxAtStartPar
Probes compatibility
&\sphinxstyletheadfamily 
\sphinxAtStartPar
Probe adapters
\\
\sphinxmidrule
\sphinxtableatstartofbodyhook
\sphinxAtStartPar
128 RX (4xus4OEM)
&
\sphinxAtStartPar
up to 128\sphinxhyphen{}element probes (linear/array/convex)
&\begin{itemize}
\item {} 
\sphinxAtStartPar
PAU (Ultrasonix Probe Adapter)

\item {} 
\sphinxAtStartPar
VPA (ATL/Philips Probe Adapter)

\item {} 
\sphinxAtStartPar
Custom Probe Adapter (on request)

\end{itemize}
\\
\sphinxhline
\sphinxAtStartPar
192 RX (6xus4OEM)
&
\sphinxAtStartPar
up to 192\sphinxhyphen{}element probes (linear/array/convex)
&\begin{itemize}
\item {} 
\sphinxAtStartPar
EPA (Ultrasonix Probe Adapter)**

\item {} 
\sphinxAtStartPar
PAU (Ultrasonix Probe Adapter)**

\item {} 
\sphinxAtStartPar
VPA (ATL/Philips Probe Adapter)**

\item {} 
\sphinxAtStartPar
Custom Probe Adapter (on request)

\end{itemize}
\\
\sphinxhline
\sphinxAtStartPar
256 RX (8xus4OEM)
&
\sphinxAtStartPar
up to 256\sphinxhyphen{}element probes (linear/array/convex) and up to 1024\sphinxhyphen{}element matrix\sphinxhyphen{}array probes
&\begin{itemize}
\item {} 
\sphinxAtStartPar
EPA (Ultrasonix Probe Adapter)

\item {} 
\sphinxAtStartPar
PAU (Ultrasonix Probe Adapter)

\item {} 
\sphinxAtStartPar
VPA (ATL/Philips Probe Adapter)

\item {} 
\sphinxAtStartPar
2D MATRIX 2372 Vermon probe

\item {} 
\sphinxAtStartPar
Custom Probe Adapter (on request)

\end{itemize}
\\
\sphinxbottomrule
\end{tabular}
\sphinxtableafterendhook\par
\sphinxattableend\end{savenotes}

\sphinxAtStartPar
\sphinxstyleemphasis{* Switching between 4x, 6x and 8x us4OEM module options can be done by
the Manufacturer only.}

\sphinxAtStartPar
\sphinxstyleemphasis{** Not easily interchangeable! If you plan to use probes above 128
elements from various manufacturers (e.g. Esaote and Philips probes) please contact us4us to find the best
solution for you.}

\sphinxAtStartPar
If you cannot find the adapter that suits your application, it is
possible to order a custom probe adapter from the us4us®. Please contact
us at \sphinxhref{mailto:support@us4us.eu}{support@us4us.eu} to discuss the options.


\chapter{Power supply connection}
\label{\detokenize{content/set-up:power-supply-connection}}
\begin{sphinxadmonition}{caution}{Caution:}
\sphinxAtStartPar
The 120\sphinxhyphen{}230VAC power socket used to power the us4R™ must be equipped with a protective earth wire. It is imperative to ensure that the electrical system provides the fire protection required for the class I devices.
\end{sphinxadmonition}

\sphinxAtStartPar
A loss of mains power during operation will result in an immediate
shutdown of the device. The \sphinxstylestrong{us4R™} will restart once the power is
restored.

\begin{sphinxadmonition}{caution}{Caution:}
\sphinxAtStartPar
To shut down the us4R™ in case of malfunction, remove the mains power cable from the electrical socket. The electrical sockets should be situated in proximity to the device and be easily accessible.
\end{sphinxadmonition}


\chapter{System setup}
\label{\detokenize{content/set-up:system-setup}}
\sphinxAtStartPar
The \sphinxstylestrong{us4R™} should be positioned so that operation is safe — i.e. on
a stable, flat surface in a place with no risk of spillage on the device
and away from the sources of interference and radiation. The external
power supply and power strip should be placed nearby. For more details
see section: \sphinxstylestrong{6 Description and general rules of use.}

\sphinxAtStartPar
The heat is dissipating by the us4R\sphinxhyphen{}3D system during normal work
conditions and may slightly increase the temperature of the device
surroundings. The ventilation holes on each side of the device must
remain uncovered to ensure free flow of air. Covering the ventilation
holes risks overheating, shutting down or damaging the \sphinxstylestrong{us4R™.} The
power and probe cables must not be strained or hang too loosely in a
manner that may lead to tripping, mechanical damage, wrenching the
cables out of the socket and/or damaging them through breaking or
cutting. The audio frequency noise is coming from the fans and sometimes
from the HV power supply during transmit during normal work conditions.
Please consult the Manufacturer guidelines in section \sphinxstylestrong{7 Manufacturer
guidelines and conditions of use.}


\section{Power switch, cables and ON/OFF button}
\label{\detokenize{content/set-up:power-switch-cables-and-on-off-button}}
\sphinxAtStartPar
The power cables connection is shown in the picture below.

\begin{figure}[htbp]
\centering
\capstart

\noindent\sphinxincludegraphics{{us4r-back}.jpeg}
\caption{The us4R™ AC power connector.}\label{\detokenize{content/set-up:id2}}\end{figure}


\section{First use}
\label{\detokenize{content/set-up:first-use}}
\begin{sphinxadmonition}{danger}{Danger:}
\sphinxAtStartPar
Never unplug the probe from the device during transmission!
This can result in damage to the transmission unit of the us4R™.
\end{sphinxadmonition}

\sphinxAtStartPar
Before first use of the \sphinxstylestrong{us4R™}, you must ensure that:
\begin{itemize}
\item {} 
\sphinxAtStartPar
the device has been set up according to Manufacturer guidelines
found in section \sphinxstylestrong{5 System setup},

\item {} 
\sphinxAtStartPar
proper probe adapter has been installed,

\item {} 
\sphinxAtStartPar
an external PC and monitor have been connected to the \sphinxstylestrong{us4R™}.

\end{itemize}

\sphinxAtStartPar
Step\sphinxhyphen{}by\sphinxhyphen{}step instruction:
\begin{enumerate}
\sphinxsetlistlabels{\arabic}{enumi}{enumii}{}{.}%
\item {} 
\sphinxAtStartPar
Connect the mains to the \sphinxstylestrong{us4R™} and turn the \sphinxstyleemphasis{Power Switch} on***.***

\item {} 
\sphinxAtStartPar
Now, turn on the device by clicking the ON/OFF button.

\item {} 
\sphinxAtStartPar
Connect the ultrasound probe.

\item {} 
\sphinxAtStartPar
Turn on the host PC

\sphinxAtStartPar
a.  Before login check the color of the LEDs on the back of the PC
\textendash{} all 4 or 8 LED indicators (from the top and bottom card)
should light up GREEN.

\end{enumerate}

\begin{figure}[htbp]
\centering
\capstart

\noindent\sphinxincludegraphics{{pcie-cables-1234}.jpeg}
\caption{The host PC: the PCIe card interface with four connected PCIe cables and the PCIe links LEDs.}\label{\detokenize{content/set-up:id3}}\end{figure}
\begin{quote}

\sphinxAtStartPar
IMPORTANT NOTE: If any of the LED indicators light up ORANGE, please reboot the PC. Keep rebooting until all LEDs are green.
\end{quote}
\begin{enumerate}
\sphinxsetlistlabels{\arabic}{enumi}{enumii}{}{.}%
\setcounter{enumi}{4}
\item {} 
\sphinxAtStartPar
Log in to the host PC:

\sphinxAtStartPar
\sphinxcode{\sphinxupquote{user: us4us}}

\sphinxAtStartPar
\sphinxcode{\sphinxupquote{password: us4us}}

\item {} 
\sphinxAtStartPar
Make sure to check that the device and its software is starting
correctly. If any errors are signaled by the device or messages
displayed on screen, proceed according to instructions.

\item {} 
\sphinxAtStartPar
Install the ARRUS package according to instructions available
\sphinxhref{https://us4useu.github.io/arrus-public/releases/current/python/content/installation/index.html\#arrus}{here}
(if it is not already installed).

\item {} 
\sphinxAtStartPar
Follow the instruction on how to run “plane wave imaging” example script available \sphinxhref{https://us4useu.github.io/arrus-public/releases/develop/python/content/examples.html}{here} (section Examples \(\rightarrow\) Plane Wave Imaging). Please remember to use the configuration file provided.

\item {} 
\sphinxAtStartPar
Once the test is over, close the image window.

\item {} 
\sphinxAtStartPar
The host PC can now be turned off by shutting down the Windows
system as normal.

\item {} 
\sphinxAtStartPar
Turn off the \sphinxstylestrong{us4R™} by pressing the ON/OFF button.

\item {} 
\sphinxAtStartPar
After 5 seconds turn off the Power Switch.

\end{enumerate}

\sphinxstepscope


\chapter{Description and general rules of use}
\label{\detokenize{content/hardware:description-and-general-rules-of-use}}\label{\detokenize{content/hardware::doc}}
\begin{sphinxadmonition}{attention}{Attention:}
\sphinxAtStartPar
The device can only be operated by users with a base knowledge of programming and fundamental PC skills. It is essential that users read the full text of the instruction manual before operating the device.
\end{sphinxadmonition}

\begin{sphinxadmonition}{attention}{Attention:}
\sphinxAtStartPar
Before first use, you must ensure that the device is complete and in good condition. Any mechanical damage, spillage stains or similar faults require servicing. Under no circumstances can a faulty or damaged network cable be used.
\end{sphinxadmonition}

\begin{sphinxadmonition}{attention}{Attention:}
\sphinxAtStartPar
Using the us4R™ out of its intended use, or any use that has not been delineated in this manual, will lower the effectiveness of measures put in place to protect the user from danger, and result in a decrease of safety levels.
\end{sphinxadmonition}

\sphinxAtStartPar
The device consists of the \sphinxstylestrong{us4R™} device, a host PC computer (picture below), and a set of cables.

\begin{figure}[htbp]
\centering
\capstart

\noindent\sphinxincludegraphics{{us4r+pc}.jpeg}
\caption{View of a complete system setup.}\label{\detokenize{content/hardware:id1}}\end{figure}

\sphinxAtStartPar
\sphinxstylestrong{As standard, an LCD monitor and ultrasound probes are not provided by the Manufacturer.}

\sphinxAtStartPar
The \sphinxstylestrong{us4R™} enables the user to simultaneously connect up to two ultrasound probes (linear/phase/convex). Two connectors (PROBE A and PROBE B) are situated at the top of the device; For matrix\sphinxhyphen{}array probe, an dedicated probe adapter is available with 4x DLM6\sphinxhyphen{}360 connectors.
Only a single probe (linear/phase/convex) probe connector is active at a time. An active probe is used to transmit and receive ultrasound signals that are acquired and processed by the system.

\sphinxAtStartPar
The active connector/probe is chosen in software.

\sphinxAtStartPar
Ultrasound echo signals from the probe are digitized and transmitted via single/dual PCIe gen3 x16 digital interface to the PC, and then further to the GPU cards. Real\sphinxhyphen{}time data processing takes place in the CPU/GPUs.

\sphinxAtStartPar
The processed data can be presented graphically on an LCD monitor. The LCD monitor is not supplied with the \sphinxstylestrong{us4R™} system and must be provided by the user.

\sphinxAtStartPar
The LCD monitor can be connected to the PC using a dedicated DisplayPort cable.

\begin{sphinxadmonition}{danger}{Danger:}
\sphinxAtStartPar
Never unplug the probe from the device during transmission!
This can result in damage to the transmit section of the us4R™ device!
\end{sphinxadmonition}

\begin{sphinxadmonition}{attention}{Attention:}
\sphinxAtStartPar
The device is not equipped with life functions monitoring or alarm systems.
The us4R™ is not designed to monitor life functions!
\end{sphinxadmonition}


\section{Inputs and outputs}
\label{\detokenize{content/hardware:inputs-and-outputs}}
\sphinxAtStartPar
The \sphinxstylestrong{us4R™} is equipped with:
\begin{itemize}
\item {} 
\sphinxAtStartPar
up to 2 or 4 probe connectors (depends on the probe adapter
installed),

\item {} 
\sphinxAtStartPar
4x or 8x PCIe ports,

\item {} 
\sphinxAtStartPar
2x digital inputs,

\item {} 
\sphinxAtStartPar
2x digital outputs,

\item {} 
\sphinxAtStartPar
1x IEC mains power input.

\end{itemize}

\begin{figure}[htbp]
\centering
\capstart

\noindent\sphinxincludegraphics{{us4r-back}.jpeg}
\caption{Back\sphinxhyphen{}side of the us4R™ device.}\label{\detokenize{content/hardware:id2}}\end{figure}

\sphinxAtStartPar
\sphinxstylestrong{PLEASE NOTE:} External devices should be connected via cables no
longer than 3m.


\section{Connecting ultrasound probes}
\label{\detokenize{content/hardware:connecting-ultrasound-probes}}
\sphinxAtStartPar
Ultrasound probes require special care, as they can be easily damaged by
an impact. The damaged transducers could have internal element
short\sphinxhyphen{}circuits or open\sphinxhyphen{}circuits, both can cause malfunction or even
breakdown of the \sphinxstylestrong{us4R™} transmit circuitry. \sphinxstylestrong{Therefore, it is vital
that the probes are handled with extreme care and defective probes are
never connected to the system.}

\sphinxAtStartPar
Probes should be disconnected from the device during the transport.

\sphinxAtStartPar
The ultrasound probe connectors are situated at the top of the device.
2D probes (linear/phase/convex) connectors are marked as \sphinxstyleemphasis{\sphinxstylestrong{PROBE A}}
and \sphinxstyleemphasis{\sphinxstylestrong{PROBE B}} on the figure below.

\sphinxAtStartPar
Please refer to section \sphinxstyleemphasis{\sphinxstylestrong{3.1 Probe Adapters}} for other probe
adapters options.

\begin{figure}[htbp]
\centering
\capstart

\noindent\sphinxincludegraphics{{us4r+pc+probe}.jpeg}
\caption{Top\sphinxhyphen{}view of the us4R™ with 2D (linear/phase/convex) PROBE A and PROBE B connectors.}\label{\detokenize{content/hardware:id3}}\end{figure}

\sphinxAtStartPar
A video instruction on how to change the probe adapter for 2D (linear/phase/convex) probes is available on our YouTube channel:

\sphinxAtStartPar
\sphinxhref{https://www.youtube.com/watch?v=v9DwhbGclBE}{\sphinxincludegraphics{{us4r-lite-change-adapter-video}.png}}

\sphinxAtStartPar
\sphinxstylestrong{PLEASE NOTE:} Only a probe prepared and configured for use with the \sphinxstylestrong{us4R™} can be connected to the device. For assistance, please contact the Manufacturer.

\begin{sphinxadmonition}{danger}{Danger:}
\sphinxAtStartPar
Using non\sphinxhyphen{}compatible or broken probes can result in damage to the transmission section of the us4R™!
Such damages are NOT covered under the warranty!
\end{sphinxadmonition}


\section{PCIe ports}
\label{\detokenize{content/hardware:pcie-ports}}
\sphinxAtStartPar
The \sphinxstylestrong{us4R™} is equipped with 4 or 8 PCIe ports on the back of the
device.

\sphinxAtStartPar
The PCIe ports are intended for connecting the system to an external
host PC using dedicated PCIe cables. The \sphinxstylestrong{us4R™} is provided with a
compatible PC controller that is already equipped with dedicated PCIe
host adapter card.


\subsection{Connecting the PCIe cables}
\label{\detokenize{content/hardware:connecting-the-pcie-cables}}
\sphinxAtStartPar
The delivered PCIe cables are marked \#1 to \#4 or \#8, to help with proper
connection of the \sphinxstylestrong{us4R™} ports numbered 1…4 or 1…8 to the
corresponding ports on the host PC side \textendash{} also numbered. \sphinxstylestrong{The proper
order of the PCIe cables is essential for device operation and cannot be
changed!}

\sphinxAtStartPar
When connecting the PCIe cables you should hear/feel “a click” to be
sure that the connector is latched properly.

\begin{figure}[htbp]
\centering
\capstart

\noindent\sphinxincludegraphics{{us4r-back+cables}.jpg}
\caption{Back panel of the us4R™ showing the PCIe connectors and properly connected cabling.}\label{\detokenize{content/hardware:us4r-back-cables}}\end{figure}

\sphinxAtStartPar
To disconnect the PCIe cables pull the green tab at the bottom of the PCI cable plug (\hyperref[\detokenize{content/hardware:us4r-back-cables}]{Fig.\@ \ref{\detokenize{content/hardware:us4r-back-cables}}}).


\subsection{Connecting host PC \& display}
\label{\detokenize{content/hardware:connecting-host-pc-display}}
\sphinxAtStartPar
The \sphinxstylestrong{us4R™} requires an external host PC with an LCD monitor to function correctly. The only way to connect the \sphinxstylestrong{us4R™} device to the PC is through the PCIe cables.

\sphinxAtStartPar
The supplied PC is equipped with one or two PCIe adapter cards \textendash{} one at the top, one at the bottom of the enclosure (\hyperref[\detokenize{content/hardware:pc-back-cables}]{Fig.\@ \ref{\detokenize{content/hardware:pc-back-cables}}}).

\sphinxAtStartPar
To disconnect PCIe cables pull the green tab at the bottom of the PCIe cable plug (\hyperref[\detokenize{content/hardware:pc-pcie-cables}]{Fig.\@ \ref{\detokenize{content/hardware:pc-pcie-cables}}}).

\begin{figure}[htbp]
\centering
\capstart

\noindent\sphinxincludegraphics{{pc-back+cables}.jpeg}
\caption{Back\sphinxhyphen{}side view of the host PC showing cables connection.}\label{\detokenize{content/hardware:pc-back-cables}}\end{figure}

\begin{figure}[htbp]
\centering
\capstart

\noindent\sphinxincludegraphics{{pc+pcie-cables}.jpeg}
\caption{The host PC: the PCIe cables \#1…\#4 connected to the bottom PCIe interface card.}\label{\detokenize{content/hardware:pc-pcie-cables}}\end{figure}

\begin{figure}[htbp]
\centering
\capstart

\noindent\sphinxincludegraphics{{pcie-cables-5678}.jpeg}
\caption{The host PC: the PCIe cables \#5…\#8 connected to the top PCIe interface card.}\label{\detokenize{content/hardware:pcie-cables-5678}}\end{figure}


\section{I/O ports}
\label{\detokenize{content/hardware:i-o-ports}}
\sphinxAtStartPar
The \sphinxstylestrong{us4R™} provides four digital I/O signals in the LVTTL 3.3V
standard available on the SMA\sphinxhyphen{}type connectors:
\begin{enumerate}
\sphinxsetlistlabels{\arabic}{enumi}{enumii}{}{.}%
\item {} 
\sphinxAtStartPar
CLOCK IN \textendash{} input of an external reference clock signal.

\item {} 
\sphinxAtStartPar
TRIG IN \textendash{} input of an external trigger signal \textendash{} can be used to
synchronize transmit events with other devices/systems.

\item {} 
\sphinxAtStartPar
CLOCK OUT \textendash{} output of an internal reference clock signal.

\item {} 
\sphinxAtStartPar
TRIG OUT \textendash{} output of an internal trigger signal \textendash{} can be used to synchronize other external devices/systems with the \sphinxstylestrong{us4R™}.

\end{enumerate}

\sphinxAtStartPar
\sphinxincludegraphics{{us4r-back}.jpeg}


\section{Setting High\sphinxhyphen{}Voltage (HV) supply for the transmitters}
\label{\detokenize{content/hardware:setting-high-voltage-hv-supply-for-the-transmitters}}
\begin{sphinxadmonition}{caution}{Caution:}
\sphinxAtStartPar
Voltages above 70VDC constitute a life hazard according to EN 61010\sphinxhyphen{}1 and great care must be takes when using the power supply at voltages above this level!
\end{sphinxadmonition}

\sphinxAtStartPar
The system TX voltage (so called HV \textendash{} High Voltage) is one of the most
crucial parameters from the system/probe safety point of view. Because
the \sphinxstylestrong{us4R™} is a research system, it enables the user to change many
TX parameters (TX scheme, PRF, TX voltage, pulse length, etc.).
\sphinxstylestrong{However, some combinations of the TX parameters can be dangerous for
the connected ultrasound probe and/or the system itself!} Therefore,
the user is fully responsible for verifying a safe set of TX parameters
that can be used with the connected probe in a given application.
\sphinxstylestrong{Application of an excessive TX voltage or power to the probe can
(will) result in damage to the system and/or the probe!}

\sphinxAtStartPar
We strongly advise to use the lowest TX voltage possible \textendash{} as low as
reasonably achievable (ALARA rule). Also, please consult us4us® and the
probe producer to get advice on the max TX voltage and power that can be
delivered to the probe.


\section{Cleaning and maintenance of the device}
\label{\detokenize{content/hardware:cleaning-and-maintenance-of-the-device}}
\sphinxAtStartPar
The \sphinxstylestrong{us4R™} device should be cleaned and disinfected according to
standard procedure. However, it is essential to take additional care not
to allow any liquids into the device, as this can lead to malfunction
and the need for servicing.

\sphinxAtStartPar
The cover of the device can be cleaned with a piece of dry cloth,
ensuring no liquids are transported inside.

\begin{sphinxadmonition}{caution}{Caution:}
\sphinxAtStartPar
You must ensure that no liquids find their way inside the device!
In case of suspected spillage or moisture inside the device,
do not connect the device to a power source or attempt to\\
turn it on, but contact technical service.
\end{sphinxadmonition}

\sphinxAtStartPar
If the device will not be used over the course of two or more days, it
should be cleaned and left protected from accidental damage, spillage or
contamination at a safe location.

\sphinxstepscope


\chapter{Software}
\label{\detokenize{content/software:software}}\label{\detokenize{content/software::doc}}

\section{Firmware update}
\label{\detokenize{content/software:firmware-update}}
\sphinxAtStartPar
NOTE: the \sphinxstylestrong{us4R™} system which is shipped to you has the latest
firmware already installed.

\sphinxAtStartPar
For the firmware update and software installation, follow the instructions available
\sphinxhref{https://us4useu.github.io/arrus-toolkit/content/installation/index.html}{here}.

\sphinxAtStartPar
Links to the ARRUS™ SDK package documentation are available
\sphinxhref{https://github.com/us4useu/arrus}{here}.

\sphinxstepscope


\chapter{Manufacturer guidelines and conditions of use}
\label{\detokenize{content/manufacturer_guidelines:manufacturer-guidelines-and-conditions-of-use}}\label{\detokenize{content/manufacturer_guidelines::doc}}

\section{Conditions of storage and transport}
\label{\detokenize{content/manufacturer_guidelines:conditions-of-storage-and-transport}}\begin{itemize}
\item {} 
\sphinxAtStartPar
temperature \sphinxhyphen{}10÷50°C,

\item {} 
\sphinxAtStartPar
relative humidity across the temperature range \textless{} 90\%,

\item {} 
\sphinxAtStartPar
atmospheric pressure 500÷1060 hPa

\end{itemize}


\section{Environmental conditions}
\label{\detokenize{content/manufacturer_guidelines:environmental-conditions}}
\sphinxAtStartPar
The \sphinxstylestrong{us4R™} is designed for use in the following conditions:
\begin{itemize}
\item {} 
\sphinxAtStartPar
temperature of environment recommended 16 ÷ 26°C, allowable 10 ÷
40°C

\item {} 
\sphinxAtStartPar
humidity across the range of allowable temperatures 30\% ÷ 85\%

\item {} 
\sphinxAtStartPar
atmospheric pressure 500÷1060 hPa

\item {} 
\sphinxAtStartPar
environment of II category surge strength (overvoltage)

\item {} 
\sphinxAtStartPar
2nd degree contamination environment

\item {} 
\sphinxAtStartPar
in closed rooms

\item {} 
\sphinxAtStartPar
up to 2000m above sea level.

\end{itemize}


\section{Manufacturer EMC recommendations}
\label{\detokenize{content/manufacturer_guidelines:manufacturer-emc-recommendations}}
\sphinxAtStartPar
\sphinxincludegraphics{{emc}.png}

\sphinxAtStartPar
The device has limited immunity to electromagnetic interference and thus
should be kept as far as possible from its sources (such as mobile
phones) during work. If additional interference signals are present and
their elimination is not possible, the registered waveforms and digital
values should be ignored.

\sphinxAtStartPar
The device has no elements sensitive to a 50Hz/60Hz magnetic field.

\sphinxAtStartPar
\sphinxstylestrong{\S{} 15.105 Information to the user.}

\sphinxAtStartPar
This equipment has been tested and found to comply with the limits for a Class A digital device, pursuant to part 15 of the FCC Rules. These limits are designed to provide reasonable protection against harmful interference when the equipment is operated in a commercial environment. This equipment generates, uses, and can radiate radio frequency energy and, if not installed and used in accordance with the instruction manual, may cause harmful interference to radio communications. Operation of this equipment in a residential area is likely to cause harmful interference in which case the user will be required to correct the interference at his own expense.


\section{Other conditions and recommendations}
\label{\detokenize{content/manufacturer_guidelines:other-conditions-and-recommendations}}
\sphinxAtStartPar
It is advised that the device operate at room temperature and at
moderate humidity. Any mechanical shocks should be avoided.

\begin{sphinxadmonition}{caution}{Caution:}
\sphinxAtStartPar
The Manufacturer recommends that you contact the service and perform a technical inspection (at the Manufacturer or remotely)
if you suspect that the device has been mechanically damaged or otherwise diverges from normal appearance.
\end{sphinxadmonition}

\sphinxAtStartPar
The sole service provider for this device is the Manufacturer, us4us Ltd.

\sphinxAtStartPar
The \sphinxstylestrong{us4R™} is an electronic device and should be disposed of
according to existing regulations.

\sphinxAtStartPar
Production of this equipment required the extraction and use of natural
resources. The equipment may contain substances that could be harmful to
the environment or human health if improperly handled at the product’s
end of life. To avoid release of such substances into the environment
and to reduce the use of natural resources, we encourage you to recycle
this product in an appropriate system that will ensure that most of the
materials are reused or recycled appropriately.

\sphinxstepscope


\chapter{Technical specification}
\label{\detokenize{content/technical_specification:technical-specification}}\label{\detokenize{content/technical_specification::doc}}

\section{Technical data}
\label{\detokenize{content/technical_specification:technical-data}}\begin{itemize}
\item {} 
\sphinxAtStartPar
Ultrasound frequencies up to 20 MHz;

\item {} 
\sphinxAtStartPar
Mains power supply 120V/60Hz, 230V/50Hz \(\pm\)10\%

\item {} 
\sphinxAtStartPar
Power consumption (average) 300W

\item {} 
\sphinxAtStartPar
Power consumption (max) 600W (expected at max power)

\item {} 
\sphinxAtStartPar
Dimensions 445 mm × 264 mm × 154* mm

\end{itemize}
\begin{quote}

\sphinxAtStartPar
\sphinxstyleemphasis{*high without probe adapter; total high may slightly differ
depending on the Probe Adapter used: +31mm (MAT2372), +37mm (EPA),
+50mm (PAU), +38mm (VPA)}
\end{quote}
\begin{itemize}
\item {} 
\sphinxAtStartPar
Weight 10.5 kg

\end{itemize}


\section{Basic Composition}
\label{\detokenize{content/technical_specification:basic-composition}}\begin{itemize}
\item {} 
\sphinxAtStartPar
the \sphinxstylestrong{us4R™} device (with probe adapter)

\item {} 
\sphinxAtStartPar
4 or 8 PCIe cable

\item {} 
\sphinxAtStartPar
1 or 2 PCIe adapter card

\item {} 
\sphinxAtStartPar
1x mains power cable

\item {} 
\sphinxAtStartPar
(optional) ultrasound probe

\item {} 
\sphinxAtStartPar
(optional) PC system controller with GPU cards

\item {} 
\sphinxAtStartPar
User Manual

\end{itemize}


\section{Detailed specification}
\label{\detokenize{content/technical_specification:detailed-specification}}

\begin{savenotes}
\sphinxatlongtablestart
\sphinxthistablewithglobalstyle
\makeatletter
  \LTleft \@totalleftmargin
  \LTright\dimexpr\columnwidth-\@totalleftmargin-\linewidth\relax plus1fill
\makeatother
\begin{longtable}{\X{15}{50}\X{35}{50}}
\sphinxtoprule
\endfirsthead

\multicolumn{2}{c}{\sphinxnorowcolor
    \makebox[0pt]{\sphinxtablecontinued{\tablename\ \thetable{} \textendash{} continued from previous page}}%
}\\
\sphinxtoprule
\endhead

\sphinxbottomrule
\multicolumn{2}{r}{\sphinxnorowcolor
    \makebox[0pt][r]{\sphinxtablecontinued{continues on next page}}%
}\\
\endfoot

\endlastfoot
\sphinxtableatstartofbodyhook

\sphinxAtStartPar
\sphinxstylestrong{Transmit}
&\\
\sphinxhline
\sphinxAtStartPar
Number of channels
&
\sphinxAtStartPar
128‑1024 (depends on configuration)
\\
\sphinxhline
\sphinxAtStartPar
Transmit frequency
&
\sphinxAtStartPar
up to 20MHz
\\
\sphinxhline
\sphinxAtStartPar
Tx time delay resolution
&
\sphinxAtStartPar
up to 5 ns (depends on the system clock)
\\
\sphinxhline
\sphinxAtStartPar
Programmable TX voltage
&
\sphinxAtStartPar
up to 180V\textasciitilde{}pp\textasciitilde{} (\(\pm\)90V)
\\
\sphinxhline
\sphinxAtStartPar
TX pulsers levels
&
\sphinxAtStartPar
3
\\
\sphinxhline
\sphinxAtStartPar
Per\sphinxhyphen{}channel programmable
&
\sphinxAtStartPar
center frequency, pulse width (pulse duty cycle), pulse length, polarity and delay
\\
\sphinxhline
\sphinxAtStartPar
Pulse repetition frequency
&
\sphinxAtStartPar
up to 100 kHz
\\
\sphinxhline
\sphinxAtStartPar
\sphinxstylestrong{Receive}
&\\
\sphinxhline
\sphinxAtStartPar
Number of channels
&
\sphinxAtStartPar
128‑256 (depends on configuration)
\\
\sphinxhline
\sphinxAtStartPar
Frequency range
&
\sphinxAtStartPar
up to 50MHz
\\
\sphinxhline
\sphinxAtStartPar
Programmable anti\sphinxhyphen{}aliasing filter (cutoff)
&
\sphinxAtStartPar
10, 15, 20, 30, 35, 50 MHz
\\
\sphinxhline
\sphinxAtStartPar
Amplifier gain
&\begin{itemize}
\item {} 
\sphinxAtStartPar
LNA with programmable gain: 24, 18, 12

\item {} 
\sphinxAtStartPar
Voltage\sphinxhyphen{}Controlled Attenuator: 40dB

\item {} 
\sphinxAtStartPar
Programmable Gain Amplifier: 24, 30 dB

\item {} 
\sphinxAtStartPar
Total signal chain gain: 54 dB (max)

\item {} 
\sphinxAtStartPar
TGC update rate 1MHz

\end{itemize}
\\
\sphinxhline
\sphinxAtStartPar
Data sampling
&
\sphinxAtStartPar
14\sphinxhyphen{}bit @ 65MSPS or 12\sphinxhyphen{}bit @ 80MSPS
\\
\sphinxhline
\sphinxAtStartPar
Raw data buffer
&
\sphinxAtStartPar
up to 128MB per channel
\\
\sphinxhline
\sphinxAtStartPar
\sphinxstylestrong{External synchronization}
&\\
\sphinxhline
\sphinxAtStartPar
Output for synchronization
&
\sphinxAtStartPar
digital, LVTTV 3.3V, 50\(\Omega\) output impedance
\\
\sphinxhline
\sphinxAtStartPar
Input for synchronization
&
\sphinxAtStartPar
digital, LVTTV 3.3V, 50\(\Omega\) input impedance
\\
\sphinxhline
\sphinxAtStartPar
Reference clock output
&
\sphinxAtStartPar
digital, LVTTV 3.3V, 50\(\Omega\) output impedance
\\
\sphinxhline
\sphinxAtStartPar
Output for synchronization
&
\sphinxAtStartPar
digital, LVTTV 3.3V, 50\(\Omega\) input impedance
\\
\sphinxhline
\sphinxAtStartPar
\sphinxstylestrong{Ultrasound probes}
&\\
\sphinxhline
\sphinxAtStartPar
Ultrasound probe connectors
&
\sphinxAtStartPar
2 connectors of the 2D (linear/phase/convex) probe or up to 4 connectors of the 2D array probe
\\
\sphinxhline
\sphinxAtStartPar
Supported probes
&\begin{itemize}
\item {} 
\sphinxAtStartPar
2D probes (linear/phase/convex): up to 192 el. from Esaote and up to 128 from ATL/Philips/Ultrasonix

\item {} 
\sphinxAtStartPar
Matrix\sphinxhyphen{}array 32x32 from Vermon (MAT2372)

\item {} 
\sphinxAtStartPar
probes with up to 1024\sphinxhyphen{}elements via a dedicated custom Probe Adapter.

\end{itemize}
\\
\sphinxhline
\sphinxAtStartPar
\sphinxstylestrong{Interface}
&\\
\sphinxhline
\sphinxAtStartPar
Data streaming interface
&
\sphinxAtStartPar
Single/Dual PCIe gen3 x16
\\
\sphinxhline
\sphinxAtStartPar
Raw data real\sphinxhyphen{}time streaming data rate (wire speed)
&
\sphinxAtStartPar
1x/2x 16 GB/s
\\
\sphinxhline
\sphinxAtStartPar
Streaming method
&
\sphinxAtStartPar
PCIe Direct Memory Access
\\
\sphinxhline
\sphinxAtStartPar
\sphinxstylestrong{Software}
&\\
\sphinxhline
\sphinxAtStartPar
Low\sphinxhyphen{}level API
&
\sphinxAtStartPar
C++ (currently RF data acquisition only); Python; Matlab®
\\
\sphinxhline
\sphinxAtStartPar
\sphinxstylestrong{Power supply}
&\\
\sphinxhline
\sphinxAtStartPar
Mains power
&
\sphinxAtStartPar
120\sphinxhyphen{}230VAC, 50Hz/60Hz
\\
\sphinxhline
\sphinxAtStartPar
Average power usage
&
\sphinxAtStartPar
300W
\\
\sphinxhline
\sphinxAtStartPar
External dimensions
&
\sphinxAtStartPar
445mm × 264mm × 154mm  (height without probe adapter; total height may slightly differ depending on the Probe Adapter used: +31mm (\sphinxstyleemphasis{MAT2372}), +37mm (EPA), +50mm (PAU), +38mm (VPA)*)
\\
\sphinxhline
\sphinxAtStartPar
Weight
&
\sphinxAtStartPar
10.5 kg
\\
\sphinxhline
\sphinxAtStartPar
\sphinxstylestrong{Requirements}
&\\
\sphinxhline
\sphinxAtStartPar
PC host
&
\sphinxAtStartPar
e.g. Lenovo Thinkstation P620
\\
\sphinxhline
\sphinxAtStartPar
CPU
&
\sphinxAtStartPar
e.g. AMD Ryzen Threadripper PRO
\\
\sphinxhline
\sphinxAtStartPar
Memory \textendash{} RAM
&
\sphinxAtStartPar
e.g. 32 GB
\\
\sphinxhline
\sphinxAtStartPar
Storage
&
\sphinxAtStartPar
e.g. 1TB NVMe PCIe
\\
\sphinxhline
\sphinxAtStartPar
Operating system
&
\sphinxAtStartPar
Microsoft Windows 10 Pro (64\sphinxhyphen{}bit) / Linux Ubuntu 20.04 or newer
\\
\sphinxhline
\sphinxAtStartPar
GPU (optional)
&
\sphinxAtStartPar
e.g. High\sphinxhyphen{}performance NVidia GPU with GPU\sphinxhyphen{}Direct support
\\
\sphinxhline
\sphinxAtStartPar
PCIe adapter card
&
\sphinxAtStartPar
Dolphinics PXH832
\\
\sphinxhline
\sphinxAtStartPar
Accessories
&
\sphinxAtStartPar
e.g. LCD monitor with DP input, USB keyboard and mouse
\\
\sphinxbottomrule
\end{longtable}
\sphinxtableafterendhook
\sphinxatlongtableend
\end{savenotes}

\sphinxstepscope


\chapter{Legend of symbols}
\label{\detokenize{content/legend_of_symbols:legend-of-symbols}}\label{\detokenize{content/legend_of_symbols::doc}}

\begin{savenotes}\sphinxattablestart
\sphinxthistablewithglobalstyle
\centering
\begin{tabulary}{\linewidth}[t]{TT}
\sphinxtoprule
\sphinxstyletheadfamily 
\sphinxAtStartPar
\sphinxstylestrong{SYMBOL}
&\sphinxstyletheadfamily 
\sphinxAtStartPar
\sphinxstylestrong{DESCRIPTION}
\\
\sphinxmidrule
\sphinxtableatstartofbodyhook
\sphinxAtStartPar
\sphinxincludegraphics[width=50\sphinxpxdimen]{{emc}.png}
&
\sphinxAtStartPar
Radiation/electromagnetic interference
\\
\sphinxhline
\sphinxAtStartPar

&
\sphinxAtStartPar

\\
\sphinxhline
\sphinxAtStartPar
\sphinxincludegraphics[width=50\sphinxpxdimen]{{warning}.png}
&
\sphinxAtStartPar
CAUTION! Consult the instruction manual before use.
\\
\sphinxhline
\sphinxAtStartPar

&
\sphinxAtStartPar

\\
\sphinxhline
\sphinxAtStartPar
\sphinxincludegraphics[width=50\sphinxpxdimen]{{onoff}.png}
&
\sphinxAtStartPar
ON/OFF switch
\\
\sphinxhline
\sphinxAtStartPar
\sphinxincludegraphics[width=50\sphinxpxdimen]{{dry}.png}
&
\sphinxAtStartPar
Keep in a dry place
\\
\sphinxhline
\sphinxAtStartPar
\sphinxincludegraphics[width=50\sphinxpxdimen]{{temperatures}.png}
&
\sphinxAtStartPar
Range of allowable temperatures.
\\
\sphinxhline
\sphinxAtStartPar
\sphinxincludegraphics[width=50\sphinxpxdimen]{{manufacture_date}.png}
&
\sphinxAtStartPar
Date of manufacture.
\\
\sphinxhline
\sphinxAtStartPar

&
\sphinxAtStartPar

\\
\sphinxhline
\sphinxAtStartPar
\sphinxincludegraphics[width=50\sphinxpxdimen]{{ce}.png}
&
\sphinxAtStartPar
CE mark confirming the completion of conformity assessment of the product.
\\
\sphinxhline
\sphinxAtStartPar

&
\sphinxAtStartPar

\\
\sphinxhline
\sphinxAtStartPar
\sphinxincludegraphics[width=50\sphinxpxdimen]{{weee}.png}
&
\sphinxAtStartPar
WEEE \textendash{} electronic device, should be disposed of according to existing regulations.
\\
\sphinxbottomrule
\end{tabulary}
\sphinxtableafterendhook\par
\sphinxattableend\end{savenotes}

\sphinxstepscope


\chapter{Current system limitations}
\label{\detokenize{content/current_limitations:current-system-limitations}}\label{\detokenize{content/current_limitations::doc}}
\sphinxAtStartPar
Here is a list current system limitations:
\begin{itemize}
\item {} 
\sphinxAtStartPar
Standard sampling frequency is 65MSPS. Sampling at 80MSPS requires a
change of system firmware.

\end{itemize}

\sphinxAtStartPar
The above limitations may be removed in future versions of the system
software/firmware.

\sphinxstepscope


\chapter{Version history}
\label{\detokenize{content/version_history:version-history}}\label{\detokenize{content/version_history::doc}}

\begin{savenotes}\sphinxattablestart
\sphinxthistablewithglobalstyle
\centering
\begin{tabular}[t]{\X{15}{55}\X{10}{55}\X{30}{55}}
\sphinxtoprule
\sphinxstyletheadfamily 
\sphinxAtStartPar
ver. / date
&\sphinxstyletheadfamily 
\sphinxAtStartPar
Author
&\sphinxstyletheadfamily 
\sphinxAtStartPar
Change description
\\
\sphinxmidrule
\sphinxtableatstartofbodyhook
\sphinxAtStartPar
5/02 MAR 2023
&
\sphinxAtStartPar
BW
&
\sphinxAtStartPar
Section \sphinxstyleemphasis{8.3 Detailed specification} updated, Section \sphinxstyleemphasis{9 Current limitations} updated
\\
\sphinxhline
\sphinxAtStartPar
4/05 DEC 2021
&
\sphinxAtStartPar
BW
&
\sphinxAtStartPar
Section \sphinxstyleemphasis{5 System setup} updated, Section \sphinxstyleemphasis{7.4 Other conditions and recommendation} extended
\\
\sphinxbottomrule
\end{tabular}
\sphinxtableafterendhook\par
\sphinxattableend\end{savenotes}



\renewcommand{\indexname}{Index}
\printindex
\end{document}